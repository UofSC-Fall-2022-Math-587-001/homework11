\documentclass[12pt]{amsart}
\usepackage{amsmath}
\usepackage{amsthm}
\usepackage{amsfonts}
\usepackage{amssymb}
\usepackage[margin=1in]{geometry}
\usepackage{stackengine}
\usepackage{hyperref}
\hypersetup{
    colorlinks=true,
    linkcolor=blue
}

\theoremstyle{definition}
\newtheorem{theorem}{Theorem}[section]
\newtheorem{lemma}[theorem]{Lemma}
\newtheorem{definition}[theorem]{Definition}
\newtheorem{corollary}[theorem]{Corollary}
\newtheorem{proposition}[theorem]{Proposition}
\newtheorem{conjecture}[theorem]{Conjecture}
\newtheorem{remark}[theorem]{Remark}
\newtheorem{example}[theorem]{Example}
\newtheorem{problem}[theorem]{Problem}
\newtheorem{notation}[theorem]{Notation}
\newtheorem{question}[theorem]{Question}
\newtheorem{caution}[theorem]{Caution}

\begin{document}

\title{Homework}

\maketitle

For this week, please answer the following questions from the text. 
I've copied the problem itself below and the question numbers for 
your convenience. 

\begin{enumerate}
	\item (3.36) This exercise asks you to use the index calculus 
		to solve a discrete logarithm problem. Let $p = 
		19079$ and $g=17$. 
	\begin{enumerate}
		\item Verify that $g^i \mod p$ is $5$-smooth for each 
			of the values $i = 3030, i = 6892,$ and $i = 
			18312$.
		\item Use your computations in part (a) and linear 
			algebra to compute the discrete logarithms 
			$\log_g(2), \log_g(3),$ and $\log_g(5)$. (Note 
			that $19078 = 2 \cdot 9539$ and that $9539$ 
			is prime.) 
		\item Verify that $19 \cdot 17^{-12400} \mod p$ is 
			$5$-smooth. 
		\item Use the values from (b) and the computation in 
			(c) to solve the discrete logarithm problem 
		\begin{displaymath}
			17^x = 19 \mod 19079
		\end{displaymath}
	\end{enumerate}
	\item (3.37) Let $p$ be an odd prime and let $a$ be an integer 
		with $p \nmid a$. 
	\begin{enumerate}
		\item Prove that $a^{(p-1)/2}$ is congruent to either $1$ 
			or $-1$ modulo $p$. 
		\item Prove that $a^{(p-1)/2}$ is congruent to $1$ modulo 
			$p$ if and only if $a$ is a quadratic residue 
			modulo $p$. (Hint: Let $g$ be a primitive root for 
			$p$ and use the fact, proven during the course of 
			proving Proposition 3.61, that $g^m$ is a quadratic 
			residue if and only if $m$ is even.) 
		\item Prove that $a^{(p-1)/2} = \left( \frac{a}{p} \right) 
			\mod p$. 
		\item Use (c) to prove Theorem 3.62(a), that is prove that 
		\begin{displaymath}
			\left( \frac{-1}{p} \right) = 
				\begin{cases} 
					1 & \text{ if } p = 1 \mod 4 \\
					-1 & \text{ if } p = 3 \mod 4 
				\end{cases}
		\end{displaymath}
	\end{enumerate} 
	\item (3.39) Let $p$ be a prime satisfying $p = 3 \mod 4$. 
	\begin{enumerate}
		\item Let $a$ be a quadratic residue modulo $p$. Prove that 
			the number 
		\begin{displaymath}
			b = a^{(p+1)/4} \mod p 
		\end{displaymath}
		has the property that $b^2 = a \mod p$. (Hint: Write $(p+1)/2$ 
		as $1 + (p-1)/2$ and use Exercise 3.37.) This gives an easy way 
		to take square roots modulo $p$ for primes that are congruent 
		to $3$ modulo $4$.
		\item Use (a) to compute the following square roots modulo $p$. 
			Be sure to check your answers. 
		\begin{enumerate}
			\item Solve $b^2 = 116 \mod 587$ 
			\item Solve $b^2 = 3217 \mod 8627$ 
			\item Solve $b^2 = 9109 \mod 10663$ 
		\end{enumerate}
	\end{enumerate}
	\item (3.40) Let $p$ be an odd prime, let $g \in \mathbb{F}_p^\times$ be 
		a primitive root, and let $h \in \mathbb{F}_p^\times$. Write 
		$p-1 = 2^s m$ with $m$ odd and $s \geq 1$, and write the binary 
		expansion of $\log_g(h)$ as 
	\begin{displaymath}
		\log_g(h) = \epsilon_0 + 2 \epsilon_1 + 4 \epsilon_2 + 8 \epsilon_3 
		+ \cdots \text{ with } \epsilon_i \in \lbrace 0,1 \rbrace. 
	\end{displaymath}
		Give an algorithm that generalizes Example 3.69 and allows you to 
		rapidly compute $\epsilon_1, \epsilon_2,\ldots, \epsilon_{s-1}$, 
		thereby proving that the first $s$ bits of the discrete logarithm 
		are insecure. You may assume you have a fast algorithm to compute 
		square roots in $\mathbb{F}_p^\times$, as provided by for example 
		by Exercise 3.39(a). (Hint: Use Example 3.69 to compute the 0th 
		bit, take the square root of either $h$ or $g^{-1}h$ and repeat.)
		$p = 3 \mod 4$
\end{enumerate}
\end{document}
